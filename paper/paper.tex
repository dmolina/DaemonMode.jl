% JuliaCon proceedings template
\documentclass{juliacon}
\setcounter{page}{1}

\begin{document}

% **************GENERATED FILE, DO NOT EDIT**************

\title{Running faster scripts with DaemonMode.jl}

\author[1, 2]{Daniel Molina}
\affil[1]{DaSCI Andalusian Institute of Data Science and Computational Intelligence, Spain}
\affil[2]{Department of Computer Science, University of Granada, Spain}

\keywords{Julia, Optimization, Tool, Script, Time-to-first plot}

\hypersetup{
pdftitle = {Running faster scripts with DaemonMode.jl},
pdfsubject = {JuliaCon 2019 Proceedings},
pdfauthor = {Daniel Molina},
pdfkeywords = {Julia, Optimization, Tool, Script, Time-to-first plot},
}



\maketitle

\begin{abstract}

  One of the disadvantages most perceived by new users is the well-known
  time-to-first plot problem, the perceived initial time for running a program
  in Julia, mainly due to the compilation phase. While it is true that this time
  is being reduced, it is still a problem to use Julia as a language to create
  scripts, specially when it is needed to load packages (like CSV or DataFrames,
  very common in data science scripts). This package reduce greatly that
  perceived time by the following mechanism: All the programs can be run by a
  process in Julia, called daemon, and there is a script, the client, that send
  the code to run through sockets to the first one. In that way, all scripts are
  run by in the same process, so they run a lot faster because the packages were
  loaded only once (when the first script is run). As result, this package allow
  scripts in julia to be run faster even small scripts or scripts using many
  packages.
\end{abstract}

\section{Introduction}

Julia is a emerging open-source programming language designed to be high-level
language programming but capable of high-performance computing
\cite{bezansonJuliaFreshApproach2017a}. It is as easy to use as Python and/or Matlab
maintaining a performance comparable to lower languages like C/C++.

Julia is able to obtain very good performance because it uses a Just-In-Time,
JIT, interpreter than compile first the code and later this compiled code is
run. This is a great difference of time, because the code that run is compiled
and optimized. It is true that there are JIT interpreter for other languages
(one of the most popular could be Pypy\footnote{\url{https://www.pypy.org/}} for
Python \cite{liAssessingOptimizingPerformance2019}), but Julia was designed
considering performance at difference of Python, whose flexibility reduce the
possibilities of JIT interpreter. A clear signal of the good performance is
that the majority of packages available in Julia have been developed in Julia,
not in a low-level language like C/C++.

The main drawback of that mechanism is the initial time that takes that process,
because not only implies to compile the main program, but also the
libraries/packages used. This time is called the \textit{first plot problem},
and it can be a great disappointment for users attracted by its greater
efficiency, since in short programs the time due to JIT can imply a considerable
delay, giving a bad impression to new users. It is true that this time is only
initial, and developers are tackling it, but for many users it is still an important problem.
In other languages, like Python, the majority of the libraries are compiled, so
they has not that problem.

The previous problem is particularly a great disadvantage when we want to use
Julia to create small scripts. Python is very used in small scripts, in
particular in Data Science. Julia could be useful for them also, specially
considering good packages/libraries like DataFrames.jl, MLJ, or Flux. However,
the initial time of these libraries is considerable, so the \textit{first plot
  problem} can be an handicap to use Julia instead of Python. One option is to
use the
\href{https://github.com/JuliaLang/PackageCompiler.jl}{PackageCompiler}\footnote{\url{https://github.com/JuliaLang/PackageCompiler.jl}}
package, but in that case all libraries should be compiled when any of them have
a new version.

The package presented in this paper,
\href{https://github.com/dmolina/DaemonMode.jl/}{DaemonMode.jl}, was designed to
solve that problem. Inspired in the daemon/client model used by previous
programs (in particular, in Emacs), this package allow users to easily run the
Julia using a server/client model connected by sockets. The idea is to have a
only Julia process, called Daemon, running the different scripts we want to run,
and using a client program that send the files to run to the daemon program.
Because the different programs are run in the same process, the libraries are
been only compiled first time they are loaded, and for next scripts they are not
compiled. The impression that it gives to the user is great reduction in time
running scripts, making Julia a clear alternative.

This paper is structured as following: In Section \ref{sec:usage}, the usage of
the package is described, both simple and advanced. In Section \ref{parallel} we
are going to . In Section \ref{sec:details}, several technical usage is
described, giving special session. 

\section{Simple and Advanced Usage}
\label{sec:usage}

\section{Parallelism usage}
\label{sec:parallel}

\section{Technical details}
\label{sec:technicals}

\section{Measure of the performance}
\label{sec:performance}


\section{Conclusions}
\label{sec:conclusions}


\input{bib.tex}

\end{document}

% Inspired by the International Journal of Computer Applications template
